% ---------------------------------------------------------------------------------------
%	PACKAGES AND OTHER DOCUMENT CONFIGURATIONS
% ---------------------------------------------------------------------------------------

\documentclass{article}

%%%%%%%%%%%%%%%%%%%%%%%%%%%%%%%%%%%%%%%%%
% Lachaise Assignment
% Structure Specification File
% Version 1.0 (26/6/2018)
%
% This template originates from:
% http://www.LaTeXTemplates.com
%
% Authors:
% Marion Lachaise & François Févotte
% Vel (vel@LaTeXTemplates.com)
%
% License:
% CC BY-NC-SA 3.0 (http://creativecommons.org/licenses/by-nc-sa/3.0/)
% 
%%%%%%%%%%%%%%%%%%%%%%%%%%%%%%%%%%%%%%%%%

%----------------------------------------------------------------------------------------
%	PACKAGES AND OTHER DOCUMENT CONFIGURATIONS
%----------------------------------------------------------------------------------------

\usepackage{amsmath,amsfonts,stmaryrd,amssymb} % Math packages
\usepackage{etoolbox} % String comparisons for formatting


% Paragraph formatting
\setlength{\parindent}{1.5em}
\setlength{\parskip}{0.6em}

\usepackage{setspace} % Line spacing and paragraph formatting
\setstretch{1.15} % Set line spacing to 1.15
\frenchspacing % Use single spaces after periods

\usepackage{enumerate} % Custom item numbers for enumerations

\usepackage[ruled]{algorithm2e} % Algorithms

\usepackage[framemethod=tikz]{mdframed} % Allows defining custom boxed/framed environments

\usepackage{listings} % File listings, with syntax highlighting
\lstset{
	basicstyle=\ttfamily, % Typeset listings in monospace font
	breaklines=true, % Enable line wrapping in listings
	% breakatwhitespace=true, % Prefer breaks at whitespace
}

%----------------------------------------------------------------------------------------
%	DOCUMENT MARGINS
%----------------------------------------------------------------------------------------

\usepackage{geometry} % Required for adjusting page dimensions and margins

\geometry{
	paper=a4paper, % Paper size, change to letterpaper for US letter size
	top=2.5cm, % Top margin
	bottom=3cm, % Bottom margin
	left=2.5cm, % Left margin
	right=2.5cm, % Right margin
	headheight=14pt, % Header height
	footskip=1.5cm, % Space from the bottom margin to the baseline of the footer
	headsep=1.2cm, % Space from the top margin to the baseline of the header
	%showframe, % Uncomment to show how the type block is set on the page
}

%----------------------------------------------------------------------------------------
%	FONTS
%----------------------------------------------------------------------------------------

\usepackage[utf8]{inputenc} % Required for inputting international characters
\usepackage[T1]{fontenc} % Output font encoding for international characters

\usepackage{XCharter} % Use the XCharter fonts

% Improve wrapping for typewriter text
\pretocmd{\ttfamily}{\hyphenchar\font=`\-}{}{}

%----------------------------------------------------------------------------------------
% 	SECTION NUMBERING
%----------------------------------------------------------------------------------------

\renewcommand{\thesection}{\arabic{section}}
\makeatletter
\renewcommand{\@seccntformat}[1]{%
\ifstrequal{#1}{section}{\thesection.\quad}{\csname the#1\endcsname\quad}%
}
\makeatother

%----------------------------------------------------------------------------------------
%	COMMAND LINE ENVIRONMENT
%----------------------------------------------------------------------------------------

% Usage:
% \begin{commandline}
%	\begin{verbatim}
%		$ ls
%		
%		Applications	Desktop	...
%	\end{verbatim}
% \end{commandline}

\mdfdefinestyle{commandline}{
	leftmargin=10pt,
	rightmargin=10pt,
	innerleftmargin=15pt,
	middlelinecolor=black!50!white,
	middlelinewidth=2pt,
	frametitlerule=false,
	backgroundcolor=black!5!white,
	frametitle={Command Line},
	frametitlefont={\normalfont\sffamily\color{white}\hspace{-1em}},
	frametitlebackgroundcolor=black!50!white,
	nobreak,
}

% Define a custom environment for command-line snapshots
\newenvironment{commandline}{
	\medskip
	\begin{mdframed}[style=commandline]
}{
	\end{mdframed}
	\medskip
}

%----------------------------------------------------------------------------------------
%	FILE CONTENTS ENVIRONMENT
%----------------------------------------------------------------------------------------

% Usage:
% \begin{file}[optional filename, defaults to "File"]
%	File contents, for example, with a listings environment
% \end{file}

\mdfdefinestyle{file}{
	innertopmargin=1.6em,
	topline=true, bottomline=true,
	leftline=true, rightline=true,
	leftmargin=0.5cm,
	rightmargin=0.5cm,
	linewidth=0.5pt,
	linecolor=black,
	singleextra ={
		\draw[fill=black!10!white](P)++(0,-1.2em)rectangle(P-|O);
		\node[anchor=north west]
		at(P-|O){\ttfamily\mdfilename};
	},
	firstextra={
		\draw[fill=black!10!white](P)++(0,-1.2em)rectangle(P-|O);
		\node[anchor=north west]
		at(P-|O){\ttfamily\mdfilename};
	},
	middleextra={},
	secondextra={},
}

% Define a custom environment for file contents
\newenvironment{file}[1][File]{ % Set the default filename to "File"
	\medskip
	\newcommand{\mdfilename}{#1}
	\begin{mdframed}[style=file]
	\footnotesize
	\setstretch{0.9}
}{
	\normalsize
	\end{mdframed}
	\medskip
}

%----------------------------------------------------------------------------------------
%	NUMBERED QUESTIONS ENVIRONMENT
%----------------------------------------------------------------------------------------

% Usage:
% \begin{question}[optional title]
%	Question contents
% \end{question}

\mdfdefinestyle{question}{
	innertopmargin=1.2\baselineskip,
	innerbottommargin=0.8\baselineskip,
	roundcorner=5pt,
	nobreak,
	singleextra={%
		\draw(P-|O)node[xshift=1em,anchor=west,fill=white,draw,rounded corners=5pt]{%
		Question \theQuestion\questionTitle};
	},
}

\newcounter{Question} % Stores the current question number that gets iterated with each new question

% Define a custom environment for numbered questions
\newenvironment{question}[1][\unskip]{
	\bigskip
	\stepcounter{Question}
	\newcommand{\questionTitle}{~#1}
	\begin{mdframed}[style=question]
}{
	\end{mdframed}
	\medskip
}

%----------------------------------------------------------------------------------------
%	WARNING TEXT ENVIRONMENT
%----------------------------------------------------------------------------------------

% Usage:
% \begin{warn}[optional title, defaults to "Warning:"]
%	Contents
% \end{warn}

\mdfdefinestyle{warning}{
	topline=false, bottomline=false,
	leftline=false, rightline=false,
	nobreak,
	singleextra={%
		\draw(P-|O)++(-0.5em,0)node(tmp1){};
		\draw(P-|O)++(0.5em,0)node(tmp2){};
		\fill[black,rotate around={45:(P-|O)}](tmp1)rectangle(tmp2);
		\node at(P-|O){\color{white}\scriptsize\bf !};
		\draw[very thick](P-|O)++(0,-1em)--(O);%--(O-|P);
	}
}

% Define a custom environment for warning text
\newenvironment{warn}[1][Warning:]{ % Set the default warning to "Warning:"
	\medskip
	\begin{mdframed}[style=warning]
		\noindent{\textbf{#1}}
}{
	\end{mdframed}
}

%----------------------------------------------------------------------------------------
%	INFORMATION ENVIRONMENT
%----------------------------------------------------------------------------------------

% Usage:
% \begin{info}[optional title, defaults to "Info:"]
% 	contents
% 	\end{info}

\mdfdefinestyle{info}{%
	topline=false, bottomline=false,
	leftline=false, rightline=false,
	nobreak,
	singleextra={%
		\fill[black](P-|O)circle[radius=0.4em];
		\node at(P-|O){\color{white}\scriptsize\bf i};
		\draw[very thick](P-|O)++(0,-0.8em)--(O);%--(O-|P);
	}
}

% Define a custom environment for information
\newenvironment{info}[1][Info:]{ % Set the default title to "Info:"
	\medskip
	\begin{mdframed}[style=info]
		\noindent{\textbf{#1}}
}{
	\end{mdframed}
}
 % Include the file specifying the document structure and custom commands

% ---------------------------------------------------------------------------------------
%	DOCUMENT INFORMATION
% ---------------------------------------------------------------------------------------

\title{COMP 474: Project Deliverable 1}
\author{Louis Choinière -- \texttt{40218808}}
\date{Concordia University -- \today}

% ---------------------------------------------------------------------------------------

\begin{document}

\maketitle % Print the title

% ---------------------------------------------------------------------------------------
%	INTRODUCTION
% ---------------------------------------------------------------------------------------

\section*{Introduction} % Unnumbered section



% ---------------------------------------------------------------------------------------

\section{Knowledge domain (D)} % Numbered section

\textbf{\textit{D}:} \texttt{Cross-Platform Hardware-Software Requirements Mapping}

The chosen problem domain involves the alignment of hardware capabilities with software demands, specifically addressing the growing complexity of matching hardware capabilities to escalating software demands. Within this, the knowledge domain \textit{D} is defined as the systematic mapping of software system requirements including operating systems, professional applications, and video games a unified, optimized hardware specification. This domain is important because, every developer provides individual "minimum" and "recommended" specs, they are often complicated and difficult to interpret to the average user.

My expertise in this domain comes from my background as a Computer Engineering student and my hands-on experience in building and repairing systems. This academic foundation provides the technical understanding of how components like the CPU, GPU, and RAM interact under load, while my personal experience as a gamer gives me "heuristic" knowledge in interpreting and compiling developer-provided specs. Furthermore, the domain is supported by a wealth of authoritative, publicly available data. This includes official hardware standards from manufacturers like Intel and NVIDIA, peer-reviewed performance benchmarks from industry-leading publications, and the standardized system requirements published by software developers.

The experiential knowledge inherent in \textit{D} is valuable to others because it transforms fragmented data into actionable intelligence. Rather than a user manually cross-referencing tens of different sets of requirements, an intelligent system in this domain can synthesize a "composite requirement profile." This allows for more informed decision-making, such as identifying where a user can save money by avoiding unnecessary overhead or where they must invest more to ensure multi-tasking stability. Ultimately, the system serves as a bridge between raw technical specifications and the practical needs of the end-user.

% ---------------------------------------------------------------------------------------

\section{Goal (G)}

\textbf{\textit{G}:} \texttt{Generate Hardware Specifications from Software Requirements}

***Rewrite*** \\
The goal of this project is to develop an intelligent system that autonomously synthesizes a single, optimized hardware specification from a user-defined list of software (including the operating system, productivity tools, and games).

% ---------------------------------------------------------------------------------------

\section{User (U)}

\textbf{\textit{U}:} \texttt{End-Users Seeking Optimal Hardware Configurations}

The primary user of this system is any individual ranging from casual consumers to professionals who want to assemble or purchase a computer but lacks the technical confidence to reconcile conflicting software requirements. By catering to those who seek to optimize their hardware investment, the system assists users in navigating the technical jargon of hardware specifications to ensure their final machine is neither underpowered for their heaviest tasks nor unnecessarily expensive due to over-provisioning.

% ---------------------------------------------------------------------------------------

\section{Implementation}

% ---------------------------------------------------------------------------------------

\subsection{Factbase (F)}

\textbf{\textit{F}:} \texttt{Hardware-Software Requirements Database}

The factbase \textit{F} consists of a comprehensive collection of hardware specifications and software requirements. This includes detailed information on CPU models, GPU capabilities, RAM types and capacities, storage options, and their respective performance benchmarks. Additionally, it encompasses the minimum and recommended system requirements for a wide range of software applications, including operating systems, productivity tools, and video games. The factbase is structured to allow for efficient querying and retrieval of relevant data based on user-defined software lists.

Here are the facts included in the factbase:

\begin{file}[factbase.clp]
\lstinputlisting{factbase.clp.temp}
\end{file}

% ---------------------------------------------------------------------------------------

\subsection{Rulebase (R)}

\textbf{\textit{R}:} \texttt{Hardware Specification Synthesis Rules}

The rulebase \textit{R} contains the logic for synthesizing hardware specifications based on the software requirements provided by the user. It includes rules for determining the necessary CPU performance, GPU capabilities, RAM capacity, and storage requirements based on the software's minimum and recommended specifications.

\begin{file}[rulebase.clp]
\lstinputlisting{rulebase.clp.temp}
\end{file}

% ---------------------------------------------------------------------------------------

\subsection{Combination (K)}

\textbf{\textit{K}:} \texttt{Hardware Specification Generation Algorithm}

\begin{file}[rulebase.clp]
\lstinputlisting{rulebase.clp.temp}
\end{file}

% ---------------------------------------------------------------------------------------

\end{document}
